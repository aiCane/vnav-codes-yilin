\section{Installing ROS}
\label{sec:a_installing_ros}

These are the actual bash code I used:

Set up keys:

\begin{minted}{bash}
gpg --keyserver 'hkp://keyserver.ubuntu.com:80' --recv-key C1CF6E31E6BADE8868B172B4F42ED6FBAB17C654
gpg --export C1CF6E31E6BADE8868B172B4F42ED6FBAB17C654 | sudo tee /usr/share/keyrings/ros.gpg > /dev/null
\end{minted}

Setup sources.list:

\begin{minted}{bash}
sudo sh -c 'echo "deb [signed-by=/usr/share/keyrings/ros.gpg] https://mirrors.ustc.edu.cn/ros/ubuntu $(lsb_release -sc) main" > /etc/apt/sources.list.d/ros-latest.list'
\end{minted}

Then we do the Desktop-Full Install:

\begin{minted}{bash}
sudo apt update
sudo apt install ros-noetic-desktop-full
\end{minted}

And we should setup the environment. By using \texttt{.bashrc} it will be done automaticly.

\begin{minted}{bash}
echo "source /opt/ros/noetic/setup.bash" >> ~/.bashrc
source ~/.bashrc
\end{minted}

For \texttt{rosdep} installation, we use:

\begin{minted}{bash}
sudo mkdir -p /etc/ros/rosdep/sources.list.d/
sudo curl -o /etc/ros/rosdep/sources.list.d/20-default.list https://mirrors.ustc.edu.cn/rosdistro/rosdep/sources.list.d/20-default.list
sed -i 's#raw.githubusercontent.com/ros/rosdistro/master#mirrors.ustc.edu.cn/rosdistro#g' /etc/ros/rosdep/sources.list.d/20-default.list

export ROSDISTRO_INDEX_URL=https://mirrors.ustc.edu.cn/rosdistro/index-v4.yaml
rosdep update

echo 'export ROSDISTRO_INDEX_URL=https://mirrors.ustc.edu.cn/rosdistro/index-v4.yaml' >> ~/.bashrc

sudo apt-get update
sudo apt install python3-rosdep python3-rosinstall python3-rosinstall-generator python3-wstool build-essential

sudo apt install python3-rosdep

sudo rosdep init
rosdep update
\end{minted}
