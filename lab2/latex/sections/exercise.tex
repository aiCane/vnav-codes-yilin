\section{Exercise}

\subsection{Basic ROS commands} % (fold)
\label{sub:basic_ros_commands}

\subsubsection{Deliverable 1 - Nodes, topics, launch files} % (fold)
\label{ssub:deliverable_1_nodes_topics_launch_files}

\textbf{1.} List the nodes running in the two-drone static scenario.

I used the \texttt{rqt\_graph} method in a new terminal, unchecked the Debug option, and downloaded the figure, shown in figure~\ref{fig:rosgraph}.

\begin{enumerate}
	\item /av1broadcaster
	\item /av2broadcaster
	\item /plots\_publisher\_node
	\item /rviz
\end{enumerate}

\textbf{2.} How could you run the two-drone static scenario without using the roslaunch command? List the commands that you would have to execute, in separate terminals, to achieve the same result.

Just simply run every node in \textit{question 1} one by one. Such as:

\begin{minted}[fontsize=\normalsize]{bash}
rosrun av1broadcaster
rosrun av2broadcaster
rosrun plots_publisher_node
rosrun rviz
\end{minted}

\textbf{3.} List the topics that each node publishes / subscribes to. What nodes are responsible for publishing the av1, av2, frames? Which topic causes rViz to plot the drone meshes?

In the figure that \texttt{rqt\_graph} method in \textit{question 1}, we can switch the \textit{Node only} to the \textit{Node/Topics (all)} mode. We can see that

\begin{itemize}
	\item Node \texttt{/tf} are resbonsible for publishing the av1 and av2;
	\item Node \texttt{/visuals} points to \texttt{/rviz} and it points to \texttt{/rosout}.
\end{itemize}

\textbf{4.} What changes if we omit \texttt{static:=True?} Why?

You can't see those two drones, because in \texttt{two\_drones.launch}\ref{ssub:a_deliverable_1_nodes_topics_launch_files} it says so. The first line says \texttt{static} is set to \texttt{false} by default. And the code generation of two drones are in a \textit{if} condition fork. If static is false, these code will not run.

% subsubsection deliverable_1_nodes_topics_launch_files (end)

% subsection basic_ros_commands (end)

\subsection{Publishing the transforms using tf} % (fold)
\label{sub:publishing_the_transforms_using_tf}

\subsubsection{Deliverable 2 - Publishing transforms} % (fold)
\label{ssub:deliverable_2_publishing_transforms}

We use this time to minus the last time, like

\begin{minted}[fontsize=\normalsize]{cpp}
double time = (ros::Time::now() - startup_time).toSec();
\end{minted}

which returns a ros::Time object. Then we use \texttt{toSec()} to transfer it to a second time, and store it into a double variable. The second step \textit{av1} and \textit{av2}'s origin, rotation and frame\_ids are set. And finally, they are published by using \texttt{sendTransform} as a method in \texttt{tf2\_ros::TransformBroadcaster}. The whole \texttt{C++} code are pasted in appendix~\ref{ssub:a_deliverable_2_publishing_transforms}.

% subsubsection deliverable_2_publishing_transforms (end)

\subsubsection{Changing the rViz fixed reference frame} % (fold)
\label{ssub:changing_the_rviz_fixed_reference_frame}

I changed the Fixed Frame in Global Options in the left from \textit{world} to \textit{av1}, and suddenly av1 stopped moving! The world starts to spit and \textit{av2} moves in a circular shape.

% subsubsection changing_the_rviz_fixed_reference_frame (end)

\subsubsection{Deliverable 3 - Looking up a transform} % (fold)
\label{ssub:deliverable_3_looking_up_a_transform}

I added the code below to transfer frames. The whole code file are pasted in appendix~\ref{ssub:a_deliverable_3_looking_up_a_transform}.

\begin{minted}[fontsize=\normalsize]{cpp}
transform = parant->tf_buffer.lookupTransform(
	ref_frame,
	dest_frame,
	ros::Time(0),
	ros::Duration(0.1)
);
\end{minted}

% subsubsection deliverable_3_looking_up_a_transform (end)

% subsection publishing_the_transforms_using_tf (end)

\subsection{Let’s do some math} % (fold)
\label{sub:let_s_do_some_math}

\subsubsection{Deliverable 4 - Mathematical derivations} % (fold)
\label{ssub:deliverable_4_mathematical_derivations}

We are required to explicitly write down all the homogeneous transformation matrices used in the process and precisely outline the logic and algebraic steps taken.

\textbf{1.} In the problem formulation~\ref{sub:a_problem_formulation}, we mentioned that AV2’s trajectory is an arc of parabola in the $x-z$ plane of the world frame. Can you prove this statement?

Since

\begin{align*}
	cos(2t)\;&=\;1\:-\:2\,sin^{2}(t),
\end{align*}

we have

\begin{align}
	z\;=\;1\:-\:2\,x^2,
\end{align}

which is an arc of parabola.

\textbf{2.} Compute $o^{1}_{2}(t)$, i.e., the position of AV2 relative to AV1’s body frame as a function of $t$.

We know that:

\begin{align}
	T_2^1\;=\;(T_1^w)^{-1}\:\cdot\:T_2^w .
\end{align}

And since

\begin{empheq}[left={\empheqlbrace}]{align*}
	R_1^w &= \begin{bmatrix}
	\cos t & -\sin t & 0 \\
	\sin t & \cos t & 0 \\
	0 & 0 & 1
	\end{bmatrix}, \\
	o_1^w &= [\cos t, \sin t, 0]^\top,
\end{empheq}

we have its homogeneous transformation as

\begin{align*}
	T_1^w = \begin{bmatrix}
	R_1^w & o_1^w \\
	\mathbf{0}^T & 1
	\end{bmatrix} = \begin{bmatrix}
	\cos t & -\sin t & 0 & \cos t \\
	\sin t & \cos t & 0 & \sin t \\
	0 & 0 & 1 & 0 \\
	0 & 0 & 0 & 1
	\end{bmatrix}.
\end{align*}

In robotics

\begin{align*}
	T^{-1} = \begin{bmatrix}
		R^\top & -R^\top \cdot o \\
		0^\top & 1
	\end{bmatrix},
\end{align*}

so,

\begin{align}
	{T_1^w}^{-1} = \begin{bmatrix}
		\cos t & -\sin t & 0 & -1 \\
		\sin t & \cos t & 0 & 0 \\
		0 & 0 & 1 & 0 \\
		0 & 0 & 0 & 1
	\end{bmatrix}.
\end{align}

Similarly, $T_2^w$'s homogeneous transformation is

\begin{align}
	T_2^w = \begin{bmatrix}
	I & o_2^w \\
	\mathbf{0}^T & 1
	\end{bmatrix} = \begin{bmatrix}
	1 & 0 & 0 & \sin t \\
	0 & 1 & 0 & 0 \\
	0 & 0 & 1 & \cos(2t) \\
	0 & 0 & 0 & 1
	\end{bmatrix}.
\end{align}

Thus,

\begin{equation*}
\begin{split}
	T_2^1 &= (T_1^w)^\top \, T_2^w \\
	      &= \begin{bmatrix}
					\cos t & -\sin t & 0 & -1 \\
					\sin t & \cos t & 0 & 0 \\
					0 & 0 & 1 & 0 \\
					0 & 0 & 0 & 1
				\end{bmatrix} \cdot \begin{bmatrix}
					1 & 0 & 0 & \sin t \\
					0 & 1 & 0 & 0 \\
					0 & 0 & 1 & \cos(2t) \\
					0 & 0 & 0 & 1
				\end{bmatrix} \\
				&= \begin{bmatrix}
					\cos t & -\sin t & 0 & \cos t\sin t-1 \\
					\sin t & \cos t & 0 & \sin^2 t \\
					0 & 0 & 1 & 0 \\
					0 & 0 & 0 & 1
				\end{bmatrix}.
\end{split}
\end{equation*}

That is,

\begin{align}
	o_2^1(t) = \begin{bmatrix}
		\cos t \cdot \sin t-1 & -\sin^2 t & \cos 2t
	\end{bmatrix}^\top.
\end{align}

\textbf{3.} Show that $o^{1}_{2}(t)$ describes a planar curve and find the equation of its plane $\Pi$.

Since

\begin{equation*}
\begin{split}
	z(t) &= \cos 2t \\
			 &= 1 - 2 \sin^2 t \\
			 &= 1 + 2 y(t),
\end{split}
\end{equation*}

we have the every point of the curve satisfying the linear equation

\begin{align}
	z - 2y = 1,
\end{align}

which is also the equation of $o_2^1$'s plane $\Pi$.

\textbf{4.} Rewrite the above trajectory explicitly using a 2D frame of reference $(x_p,y_p)$ on the plane found before. Try to ensure that the curve is centered at the origin of this 2D frame and that $x_p$, $y_p$ are axes of symmetry for the curve.

The ``HINT'' says the new origen of the frame is

\begin{align}
p' = (-1, -\frac{1}{2}, 0)^\top.
\end{align}

So we have

\begin{equation}
\begin{split}
	o(t) &= o_1^1(t) - p' \\
			 &= \begin{bmatrix}
				 \cos t \cdot \sin t-1 \\
				 -\sin^2 t \\
				 \cos 2t
			 \end{bmatrix} - \begin{bmatrix}
			 	 -1 \\
			 	 -\frac{1}{2} \\
			 	 0
			 \end{bmatrix} \\
			 &= \begin{bmatrix}
				 \cos t \cdot \sin t \\
				 -\sin^2 t + \frac{1}{2} \\
				 \cos 2t
			 \end{bmatrix}
\end{split}
\end{equation}

In frame AV1 we define

\begin{empheq}[left={\empheqlbrace}]{align*}
	x_p &= \begin{bmatrix}
	1 & 0 & 0
	\end{bmatrix}^\top, \\
	y_p &= \begin{bmatrix}
	0 & \frac{\sqrt{5}}{5} & \frac{2\sqrt{5}}{5}
	\end{bmatrix}^\top, \\
	z_p &= \begin{bmatrix}
	0 & -\frac{2\sqrt{5}}{5} & \frac{\sqrt{5}}{5}
	\end{bmatrix}^\top.
\end{empheq}

So,

\begin{align}
	R_p^1 = \begin{bmatrix}
		x_p^1 & y_p^1 & z_p^1
	\end{bmatrix} = \begin{bmatrix}
		1 & 0 & 0 \\
		0 & \frac{\sqrt{5}}{5} & -\frac{2\sqrt{5}}{5} \\
		0 & -\frac{2\sqrt{5}}{5} & \frac{\sqrt{5}}{5}
	\end{bmatrix}.
\end{align}

Therefore,
\begin{equation}
\begin{split}
	o_2^p(t) &= R_p^1 \cdot o(t) \\
					 &= \begin{bmatrix}
						 1 & 0 & 0 \\
						 0 & \frac{\sqrt{5}}{5} & -\frac{2\sqrt{5}}{5} \\
						 0 & -\frac{2\sqrt{5}}{5} & \frac{\sqrt{5}}{5}
					 \end{bmatrix} \cdot \begin{bmatrix}
						 \cos t \cdot \sin t \\
						 -\sin^2 t + \frac{1}{2} \\
						 \cos 2t
					 \end{bmatrix} \\
					 &= \begin{bmatrix}
					 	 \frac{1}{2}\sin 2t \\
					 	 -\frac{\sqrt{5}}{2}\cos 2t \\
					 	 0
					 \end{bmatrix}.
\end{split}
\end{equation}

\textbf{5.} Using the expression of $o^{p}_{2}(t)$, prove that the trajectory of AV2 relative to AV1 is an ellipse and compute the lengths of its semi-axes.

In \textbf{question 4}, we reached that

\begin{align*}
	o_2^p(t) = \begin{bmatrix}
		\frac{1}{2}\sin 2t \\
		-\frac{\sqrt{5}}{2}\cos 2t \\
		0
	\end{bmatrix},
\end{align*}

Obviously, that is

\begin{align}
	\frac{{x_p}^2}{\frac{1}{4}} + \frac{{y_p}^2}{\frac{5}{4}} = 1,
\end{align}

and

\begin{align}
	a = \frac{1}{2}, b = \frac{\sqrt{5}}{2}
\end{align}

% subsubsection deliverable_4_mathematical_derivations (end)

\subsubsection{Deliverable 5 - More properties of quaternions} % (fold)
\label{ssub:deliverable_5_more_properties_of_quaternions}

In the lecture notes, we have defined two linear maps $\Omega\_1$: $\mathbb{R}^4 \to \mathbb{R}^{4 \times 4}$, and $\Omega\_2$: $\mathbb{R}^4 \to \mathbb{R}^{4 \times 4}$, such that for any $q \in \mathbb{R}^4$, we have:

\begin{align*}
	\Omega_1(q) &= \begin{bmatrix}
		q_4 & -q_3 & q_2 & q_1 \\
		q_3 & q_4 & -q_1 & q_2 \\
		-q_2 & q_1 & q_4 & q_3 \\
		-q_1 & -q_2 & -q_3 & q_4
	\end{bmatrix}, \\
	\Omega_2(q) &= \begin{bmatrix}
		q_4 & q_3 & -q_2 & q_1 \\
		-q_3 & q_4 & q_1 & q_2 \\
		q_2 & -q_1 & q_4 & q_3 \\
		-q_1 & -q_2 & -q_3 & q_4
	\end{bmatrix}.
\end{align*}

The product between any two unit quaternions can then be explicitly computed as:

\begin{align*}
q_a \otimes q_b = \Omega_1(q_a)q_b = \Omega_2(q_b)q_a.
\end{align*}

\textbf{1.} For any unit quaternion $q$, both $\Omega_1(q)$ and $\Omega_2(q)$ are orthogonal matrices, i.e.,

\begin{align*}
\Omega_1(q)^T\Omega_1(q) &= \Omega_1(q)\Omega_1(q)^T = I_4, \\
\Omega_2(q)^T\Omega_2(q) &= \Omega_2(q)\Omega_2(q)^T = I_4.
\end{align*}

Intuitively, what is the reason that both \(\Omega_1(q)\) and \(\Omega_2(q)\) must be orthogonal?

To make it easy, it can directly calculated. Only when $q$ is a unit quaternion, we have the result of

\begin{align*}
	\Omega_1(q)^\top \Omega_1(q) = {||q||}^2 I.
\end{align*}

And for $\Omega_2(q)$, the result would be the same.

2. For any unit quaternion $q$, both $\Omega_1(q)$ and $\Omega_2(q)$ convert $q$ to be the unit quaternion that corresponds to the 3D identity rotation, i.e.,

\begin{align*}
\Omega_1(q)^Tq = \Omega_2(q)^Tq = [0, 0, 0, 1]^T.
\end{align*}

I cannot tell why, but it can alwo be calculated:

\begin{align*}
	\begin{bmatrix}
		q_4 & -q_3 & q_2 & q_1 \\
		q_3 & q_4 & -q_1 & q_2 \\
		-q_2 & q_1 & q_4 & q_3 \\
		-q_1 & -q_2 & -q_3 & q_4
	\end{bmatrix} \begin{bmatrix}
		q_1 \\
		q_2 \\
		q_3 \\
		q_4
	\end{bmatrix} = \begin{bmatrix}
		0 \\
		0 \\
		0 \\
		1
	\end{bmatrix}
\end{align*}

3. For any two vectors $x, y \in \mathbb{R}^4$, show the two linear operators commute, i.e.,

\begin{align*}
\Omega_1(x)\Omega_2(y) &= \Omega_2(y)\Omega_1(x), \\
\Omega_1(x)\Omega_2(y)^T &= \Omega_2(y)^T\Omega_1(x).
\end{align*}

Since,

\begin{equation*}
\begin{split}
	\Omega_1(x)(\Omega_2(y)v) &= x \otimes (v \otimes y) \\
	\Omega_2(y)(\Omega_1(x)v) &= (x \otimes v) \otimes y,
\end{split}
\end{equation*}

for every $v \in \mathbb{R}^4$ it should be right.

That is $\Omega_1(x)\Omega_2(y) = \Omega_2(y)\Omega_1(x)$. Similarly, for $\Omega_1(x)\Omega_2(y)^T = \Omega_2(y)^T\Omega_1(x)$, the result would be the same.

% subsubsection deliverable_5_more_properties_of_quaternions (end)

\subsubsection{[Optional] Deliverable 6 - Intrinsic vs Extrinsic rotations} % (fold)
\label{ssub:_optional_deliverable_6_intrinsic_vs_extrinsic_rotations}

In OUC class this is no ``optional''. Though I'm very interested in it, I have no time to finish another deliverable.

% subsubsection _optional_deliverable_6_intrinsic_vs_extrinsic_rotations (end)

% subsection let_s_do_some_math (end)
