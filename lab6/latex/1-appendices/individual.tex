\section{Individual}

\subsection*{Can you do better than Nister?}
\label{sub:can_you_do_better_than_nister_}

Nister’s method is a minimal solver since it uses 5 point correspondences to compute the 5 degrees of freedom that define the relative pose (up to scale) between the two cameras (recall: each point induces a scalar equation). In the presence of external information (e.g., data from other sensors), we may be able use less point correspondences to compute the relative pose.

Consider a drone flying in an unknown environment, and equipped with a camera and an Inertial Measurement Unit (IMU). We want to use the feature correspondences extracted in the images captured at two consecutive time instants \(t_1\) and \(t_2\) to estimate the relative pose (up to scale) between the pose at time \(t_1\) and the pose at time \(t_2\). Besides the camera, we can use the IMU (and in particular the gyroscopes in the IMU) to estimate the relative rotation between the pose of the camera at time \(t_1\) and \(t_2\).
