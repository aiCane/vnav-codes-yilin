\section{Individual}

\subsection{Deliverable - Single-segment trajectory optimization}

\textit{1.} Sol.

Since $P(t) = p_1t$, and $P(1) = 1$, we got

\begin{align*}
  &P(1) = 1 = p_1 \cdot 1  \\ 
  \Rightarrow \quad &p_1 = 1 \\
  \Rightarrow \quad &P(t) = t.
\end{align*}

Since we have only one solution, this is the optimal solution. Here, $P^{(1)}(t) = t' = 1$, put it to aim function, we have

\begin{equation*}
  \int_0^1(P^{(1)}(t))^2dt = \int_0^11^2dt = 1.
\end{equation*}

That is, the optimal solution is $P(t) = t$, and the value of the cost function is $1$.

\textit{2.} Sol.

\textit{(a)} We have

\begin{align*}
  \int_0^1(p^{(1)}(t))^2dt &= \int_0^1((p_2t^2 + p_1t)')^2dt \\
  &= \int_0^1(2p_2t + p_1)^2dt \\
  &= \int_0^1(4{p_2}^2t^2 + 4p_1p_2t + {p_1}^2)dt \\
  &= [\frac{4}{3}{p_2}^2t^3 + 2p_1p_2t^2 + {p_1}^2t]_0^1 \\
  \text{so, }\int_0^1(p^{(1)}(t))^2dt &= \frac{4}{3}{p_2}^2 + 2p_1p_2 + {p_1}^2.
\end{align*}

Since $\mathbf{p} = [p_1, p_2]^\top$, we got the symmetric matrix $\mathbf{Q}$:

\begin{align*}
  \mathbf{Q} &= \begin{bmatrix}
    1 & 1 \\
    1 & \frac{4}{3}
  \end{bmatrix}, \\
  s.t. \quad \int_0^1(p^{(1)}(t))^2dt &= \mathbf{p}^\top \mathbf{Q} \mathbf{p}.
\end{align*}

\textit{(b)} Since $P(t) = p_2 t^2 + p_1 t = 1$, we have $p_2 + p_1 = 1$. Therefore, it can be written as $\mathbf{A} \mathbf{p} = \mathbf{b}$, where:

\begin{equation*}
  \mathbf{A} = [1, 1], \quad \mathbf{b} = 1.
\end{equation*}

\textit{(c)} For $P(t) = p_2 t^2 + p_1 t = 1$, we have the problem:

\begin{equation*}
  \min_{\mathbf{p}} \mathbf{p}^\top \mathbf{Q} \mathbf{p} \quad \text{s.t.} \quad \mathbf{A} \mathbf{p} = \mathbf{b},
\end{equation*}

where $\mathbf{Q} = \begin{bmatrix} 1 & 1 \\ 1 & \frac{4}{3} \end{bmatrix}, \mathbf{A} = [1, 1], \mathbf{b} = 1$.

We import Lagrangian Function:

\begin{equation}
  \mathcal{L}(\mathbf{p}, \lambda) = \mathbf{p}^\top \mathbf{Q} \mathbf{p} - \lambda (\mathbf{A} \mathbf{p} - \mathbf{b}).
\end{equation}

It has a KKT case: the partial derivatives with respect to $\mathbf{p}$ is zero:

\begin{equation*}
  \frac{\partial \mathcal{L}}{\partial \mathbf{p}} = 2 \mathbf{Q} \mathbf{p} - \mathbf{A}^\top \lambda = 0 \quad \Rightarrow \quad 2 \mathbf{Q} \mathbf{p} = \mathbf{A}^\top \lambda.
\end{equation*}

Since $\mathbf{A} \mathbf{p} = \mathbf{b}$,

\begin{equation*}
  \mathbf{p} = \mathbf{Q}^{-1} \mathbf{A}^\top \left( \mathbf{A} \mathbf{Q}^{-1} \mathbf{A}^\top \right)^{-1} \mathbf{b}.
\end{equation*}

We got $p_1 = 1$, $p_2 = 0$, here the function is:

\begin{equation*}
  \mathbf{p}^\top \mathbf{Q} \mathbf{p} = [1, 0] \begin{bmatrix} 1 & 1 \\ 1 & \frac{4}{3} \end{bmatrix} \begin{bmatrix} 1 \\ 0 \end{bmatrix} = 1.
\end{equation*}

The result is: the optimal solution is $P(t) = t$, with the value of cost is $1$.

\textit{3.} Sol.

\textit{(a)} Similar to \textit{2.(a)(b)}, we have

\begin{align*}
  \int_0^1 \left( P^{(1)}(t) \right)^2 dt &= \int_0^1 \left( p_1 + 2p_2 t + 3p_3 t^2 \right)^2 dt \\
  &= p_1^2 + 2p_1p_2 + 2p_1p_3 + \frac{4}{3}p_2^2 + 3p_2p_3 + \frac{9}{5}p_3^2.
\end{align*}

Therefore its symmetric matrix $\mathbf{Q} \in \mathbb{S}^3$ is:

\begin{equation*}
  \mathbf{Q} = \begin{bmatrix}
  1 & 1 & 1 \\
  1 & \frac{4}{3} & \frac{3}{2} \\
  1 & \frac{3}{2} & \frac{9}{5}
  \end{bmatrix}.
\end{equation*}

For $P(1) = 1$, there is $p_1 + p_2 + p_3 = 1$, so

\begin{equation*}
  \mathbf{A} = \begin{bmatrix} 1 & 1 & 1 \end{bmatrix}, \quad b = 1.
\end{equation*}

\textit{(b)} Similar to \textit{2.(c)}, we got

\begin{equation*}
\begin{cases}
  p_1 + p_2 + p_3 &= 1, \\
  p_1 + \frac{4}{3} p_2 + \frac{3}{2} p_3 &= 1, \\
  p_1 + \frac{3}{2} p_2 + \frac{9}{5} p_3 &= 1.
\end{cases}
\end{equation*}

That is $P(t) = t$, with $p_1 = 1$, $p_2 = 0$, $p_3 = 0$.

\textit{4.} Sol.

\textit{(a)} The constraints are:

\begin{equation*}
  P(1) = p_1 + p_2 = 1, \quad P^{(1)}(1) = p_1 + 2p_2 = -2.
\end{equation*}

Solving this linear system yields:

\begin{equation*}
  p_1 = 4, \quad p_2 = -3.
\end{equation*}

Thus, the optimal solution is:

\begin{equation*}
  P(t) = 4t - 3t^2.
\end{equation*}

Compute the optimal cost:
\begin{align*}
  \int_0^1 (P^{(1)}(t))^2 dt &= \int_0^1 (4 - 6t)^2 dt \\
  &= \int_0^1 (16 - 48t + 36t^2) dt \\
  &= \left[ 16t - 24t^2 + 12t^3 \right]_0^1 \\
  &= 4.
\end{align*}

That is, the optimal solution is $P(t) = 4t - 3t^2$, and the optimal cost is $4$.

\textit{(b)} Similarly, the constraints are:

\begin{equation*}
  P(1) = p_1 + p_2 + p_3 = 1, \quad P^{(1)}(1) = p_1 + 2p_2 + 3p_3 = -2.
\end{equation*}

Solving these gives:

\begin{equation*}
  p_1 = 4 + p_3, \quad p_2 = -3 - 2p_3.
\end{equation*}

The objective function is $\mathbf{p}^\top \mathbf{Q} \mathbf{p}$, where $\mathbf{Q}$ is:

\begin{equation*}
  \mathbf{Q} = \begin{bmatrix}
  1 & 1 & 1 \\
  1 & \frac{4}{3} & \frac{3}{2} \\
  1 & \frac{3}{2} & \frac{9}{5}
  \end{bmatrix}.
\end{equation*}

Substituting the parameterization yields a quadratic function in $p_3$:

\begin{equation*}
  f(p_3) = \frac{2}{15} p_3^2 + p_3 + 4.
\end{equation*}

Setting the derivative to zero:

\begin{equation*}
  \frac{df}{dp_3} = \frac{4}{15} p_3 + 1 = 0 \implies p_3 = -\frac{15}{4}.
\end{equation*}

Substituting back:

\begin{equation*}
  p_1 = 4 - \frac{15}{4} = \frac{1}{4}, \quad p_2 = -3 - 2\left(-\frac{15}{4}\right) = \frac{9}{2}.
\end{equation*}

Thus, the optimal solution is:

\begin{equation*}
  P(t) = \frac{1}{4} t + \frac{9}{2} t^2 - \frac{15}{4} t^3.
\end{equation*}

Compute the optimal cost:

\begin{equation*}
  \int_0^1 (P^{(1)}(t))^2 dt = \frac{17}{8}.
\end{equation*}

That is, the optimal solution is $P(t) = \frac{1}{4} t + \frac{9}{2} t^2 - \frac{15}{4} t^3$, and the optimal cost is $\frac{17}{8}$.

\subsection{Deliverable - Multi-segment trajectory optimization}

\textit{1.} Sol.

We minimize the squared $r$-th derivative (snap, $r=4$). By the Euler--Lagrange equation, the optimal trajectory is a polynomial of degree $2r-1 = 7$ on each segment. For $k$ segments, the total number of unknown coefficients is $\mathbf{8k}$.

\noindent Constraints:

Waypoint Constraints: These enforce the trajectory to pass through the given waypoints.

\begin{itemize}
  \item Start point $t_0$: $x_1(t_0) = P_0$ (1 constraint).
  \item End point $t_k$: $x_k(t_k) = P_k$ (1 constraint).
  \item Each intermediate point $t_i$ ($i=1,\dots,k-1$) gives 2 constraints: $x_i(t_i) = P_i$ and $x_{i+1}(t_i) = P_i$.
\end{itemize}

Total waypoint constraints: $1 + 1 + 2(k-1) = \mathbf{2k}$.

Free Derivative Constraints (Continuity): To ensure smoothness, at each intermediate point we require continuity of derivatives up to order $2r-2 = 6$ (i.e., velocity, acceleration, jerk, snap, crackle, pop). This gives $6$ constraints per intermediate point. Total free derivative constraints: $\mathbf{6(k-1)}$.

Fixed Derivative Constraints: Boundary conditions at the start and end of the full trajectory. The total constraints must equal the total unknowns: $8k$. Currently we have $2k + 6(k-1) = 8k - 6$ constraints. Thus, we need $\mathbf{6}$ additional fixed derivative constraints (typically setting velocity, acceleration, and jerk to zero or specified values at both boundaries).

That is:

\begin{equation}
\begin{cases}
  \text{Waypoint constraints:} &\mathbf{2k}, \\
  \text{Free derivative constraints:} &\mathbf{6(k-1)}, \\
  \text{Fixed derivative constraints:} &\mathbf{6}.
\end{cases}
\end{equation}

\textit{2.} Sol.

Unknowns: Each segment is a polynomial of degree $2r-1$, so there are $2r$ coefficients per segment. Total unknowns: $\mathbf{2rk}$.

Waypoint Constraints: As above, each segment must match the prescribed start and end positions. Total: $\mathbf{2k}$.

Free Derivative Constraints: At each intermediate point, continuity of derivatives from order $1$ to $2r-2$ is required. Each intermediate point contributes $2r-2$ constraints. With $k-1$ intermediate points, total: $\mathbf{(2r-2)(k-1)}$.

Fixed Derivative Constraints: The remaining constraints needed to match the number of unknowns:

\begin{align*}
  \text{Fixed} &= 2rk - \bigl[2k + (2r-2)(k-1)\bigr] \\
               &= 2rk - \bigl[2k + 2rk - 2k - 2r + 2\bigr] \\
               &= \mathbf{2r - 2}.
\end{align*}

These are typically imposed as boundary conditions on derivatives at the start and end points.

\noindent Summary for general $r$, $k$ segments:

\begin{equation}
\begin{cases}
  \text{Waypoint constraints:} &\mathbf{2k}, \\
  \text{Free derivative constraints:} &\mathbf{(2r-2)(k-1)}, \\
  \text{Fixed derivative constraints:} &\mathbf{2r-2}.
\end{cases}
\end{equation}
